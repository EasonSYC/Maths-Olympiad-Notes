%!TEX TX-program = xelatex
\documentclass[8pt]{article}

\usepackage{ctex}
\usepackage{graphicx}
\usepackage{enumitem}
\usepackage{geometry}
\usepackage{amsmath}
\usepackage{amssymb}
\usepackage{amsfonts}
\usepackage{tikz}
\usepackage{extarrows}
\usetikzlibrary{positioning}
\usepackage{xcolor}

\graphicspath{ {./images/} }

\title{\S 11 不等式中的恒等变形'}
\author{高一(6)班\ 邵亦成\ 26号}
\date{2021年12月18日}

\geometry{a4paper, scale=0.85}

\begin{document}

	\maketitle

	\begin{enumerate}
		\item 对于正实数$a_1, a_2, \cdots, a_n$, 证明: $$\sum_{i<j} \frac{a_i a_j}{a_i + a_j} \leq \frac{n}{2(a_1 + a_2 + \cdots + a_n)} \sum_{i<j} a_i a_j.$$
			~\\

			有

			$$2\sum_{i<j} a_i a_j = \left(\sum a_i\right)^2 - \left(\sum a_i^2\right),$$

			故即证

			$$\sum_{i<j} \frac{a_i a_j}{a_i + a_j} \leq \frac{1}{4} n\sum a_i - \frac{n\sum a_i^2}{4\sum a_i}.$$

			考虑通过调和平均数证明: 由

			$$\sum_{i<j} \frac{a_i a_j}{a_i + a_j} \leq \sum \frac{a_i + a_j}{4} = \frac{n-1}{4} \sum a_i,$$

			只需证

			$$\frac{n-1}{4} \sum a_i \leq \frac{1}{4} n\sum a_i - \frac{n\sum a_i^2}{4 \sum a_i},$$

			即证

			$$\sum a_i \geq n \frac{\sum a_i^2}{\sum a_i},$$

			并非恒成立.

			考虑化简不等式: 有

			\begin{align*}
				\sum \frac{a_i a_j}{a_i + a_j} &= \sum \frac{a_i + a_j}{4} - \frac{(a_i - a_j)^2}{4(a_i + a_j)}\\
				&= \frac{n-1}{4} \sum a_i - \sum \frac{(a_i - a_j)^2}{4(a_i + a_j)},
			\end{align*}

			故即证

			$$\frac{n-1}{4} \sum a_i - \sum \frac{(a_i - a_j)^2}{4(a_i + a_j)} \leq \frac{1}{4} n\sum a_i - \frac{n \sum a_i^2}{4 \sum a_i},$$

			即证

			$$a_i \geq \frac{n \sum a_i^2}{4 \sum a_i} - \sum \frac{(a_i - a_j)^2}{4(a_i + a_j)},$$

			即证

			$$-\sum \frac{(a_i - a_j)^2}{a_i + a_j} \leq \sum a_i - \frac{n \sum a_i^2}{\sum a_i} = \frac{\left(\sum a_i\right)^2 - n \sum a_i^2}{\sum a_i} = - \frac{\sum (a_i - a_j)^2}{\sum a_i},$$

			即证

			$$\frac{\sum (a_i - a_j)^2}{\sum a_i} \leq \sum \frac{(a_i - a_j)^2}{a_i + a_j},$$

			显然成立.

		~\\

		\item 设$n$为给定的正整数, $x_1, x_2, \cdots, x_n$为正实数, 证明: $$\sum_{i=1}^{n} x_i \left[1-\left(\sum_{j=1}^{i} x_j\right)^2\right] \leq \frac{2}{3}.$$
			~\\

			记

			$$S_i = \sum_{j=1}^{i} x_j,$$

			令$S_0 = 0,$ 则有$0< S_1 < S_2 < \cdots < S_n$,

			即证

			$$\sum_{i=1}^{n} \left(S_i - S_{i-1}\right) \left(1 - S_i\right)^2 \leq \frac{2}{3},$$

			即证

			$$S_n - \sum_{i=1}^{n} \left(S_i - S_{i-1}\right) S_i^2 \leq \frac{2}{3},$$

			即证

			$$\sum_{i=1}^{n} \left(S_i - S_{i-1}\right) S_i^2 \geq S_n - \frac{2}{3}.$$

			又

			$$S_i^2 \geq \frac{S_i^2 + S_{i-1} S_i + S_{i-1}^2}{3},$$

			故

			$$\sum_{i=1}^{n} \left(S_i - S_{i-1}\right) S_i^2 \geq \sum_{i=1}^{n} \frac{S_i^3 - S_{i-1}^3}{3}=\frac{1}{3} S_n^3,$$

			只需证

			$$\frac{1}{3} S_n^3 \geq S_n - \frac{2}{3},$$

			成立.

		~\\

		\textbf{Abel 分步求和公式}: 令$S_k = \displaystyle\sum_{i=1}^{k} a_i$, 则$$\sum_{k=1}^{n} a_k b_k = \sum_{k=1}^{n-1} S_k \left(b_k - b_{k+1}\right) + b_n S_n.$$

		不妨令$S_0 = 0$, 有

		\begin{align*}
			\sum_{k=1}^{n} a_k b_k &= \sum_{k=1}^{n} \left(S_k - S_{k-1}\right) b_k\\
			&= \sum_{k=1}^{n-1} S_l \left(b_k - b_{k+1}\right) + S_0 b_0 + S_n b_n.
		\end{align*}

		~\\

		\item 设$b_1 \geq b_2 \geq \cdots \geq b_n > 0$, $\displaystyle m\leq \sum_{k=1}^{t} a_k \leq M (t=1, 2, \cdots, n)$, 求证: $$b_1 m \leq \sum_{k=1}^{n} a_k b_k \leq b_1 M.$$
			~\\

			\begin{align*}
				\sum_{k=1}^{n} a_k b_k &= \sum_{k=1}^{n-1} S_k \left(b_k - b_{k+1}\right) + S_n b_n\\
				&\leq \sum_{k=1}^{n-1} M\left(b_k - b_{k+1}\right) + M\cdot b_n\\
				&= M \cdot b_1,
			\end{align*}

			左侧同理.

		~\\

		\item 已知$a_1, a_2, \cdots, a_n$和$b_1, b_2, \cdots, b_n$都是实数. 证明: 使得对任何满足$x_1 \leq x_2 \leq \cdots \leq x_n$的实数, 不等式$$\sum_{i=1}^{n} a_i x_i \leq \sum_{i=1}^{n} b_i x_i$$恒成立的充要条件是$$\sum_{i=1}^{k} a_i \geq \sum_{i=1}^{k} b_i (k=1, 2, \cdots, n-1)$$且$$\sum_{i=1}^{n} a_i = \sum_{i=1}^{n} b_i.$$
			~\\

			记

			$$\sum_{i=1}^{k} (a_k - b_k) = S_k,$$

			有

			\begin{align*}
			\sum_{i=1}^{n} a_i x_i - \sum_{i=1}^{n} b_i x+i &= \sum_{i=1}^{n} (a_i - b_i) x_i\\
			&= \sum_{i=1}^{n-1} S_i (x_i - x_{i+1}) + S_n x_n.
			\end{align*}

			故即证

			$$\forall x_1 \leq x_2 \leq \cdots \leq x_n, \sum_{i=1}^{n} S_i (x_i - x_{i+1}) + S_n x_m \leq 0 \Leftrightarrow S_1, s_2, \cdots, S_{n-1} \geq 0, S_n = 0.$$

			$\Leftarrow$ 显然成立.

			$\Rightarrow$ 令$x_1 = \cdots = x_n = 1$, $S_n \leq 0$; 令$x_1 = x_2 = \cdots = x_n = -1$, $S_n \geq 0$. 故$S_n = 0$.

			令$x_1 = \cdots = x_k = 0, x_{k=1} = \cdots = x_n = 1$, 得$S_k (x_k - x_{k+1}) \leq 0, S_k \geq 0.$

		~\\

		\item 给定正整数$n\geq 2$, 正数数列$a_1, a_2, \cdots, a_n$满足$a_k \geq a_1 + a_2 + \cdots + a_{k-1} (k=2, 3, \cdots, n)$, 求$\dfrac{a_1}{a_2} + \dfrac{a_2}{a_3} + \cdots + \dfrac{a_{n-1}}{a_n}$的最大值, 并求取得最大值的条件.
			~\\

			记

			$$S_k = a_1 + \cdots + a_{k-1}, S_0 = 0,$$

			有$a_k \geq S_{k-1}.$

			\begin{align*}
				T=\text{ 原式 } &= \sum_{k=1}^{n-1} \frac{S_k - S_{k-1}}{a_{k+1}}\\
				&= \sum_{k=1}^{n-2} S_k \left(\frac{1}{a_{k+1}} - \frac{1}{a_{k+2}}\right) + \frac{S_{n-1}}{a_n}\\
				&\leq \sum_{k=1}^{n-2} a_{k+1} \left(\frac{1}{a_{k+1}} - \frac{1}{a_{k+2}}\right) + \frac{S_{n-1}}{a_n}\\
				&= (n-2) - \sum_{k=1}^{n-2} \frac{a_{k+1}}{a_{k+2}} + \frac{S_{n-1}}{a_n}\\
				&\leq m-2 - \left(T-\frac{a_1}{a_2} \right) + 1,
			\end{align*}

			故

			$$2T \leq n-1 + \frac{a_1}{a_2} \leq n, T\leq \frac{n}{2},$$

			等号成立条件: $$a_n = 2^{n-2} a_1.$$

		~\\

		\item 设$a_i, b_i > 0 (1 \leq i \leq n+1), b_{i+1} - b_i \geq \delta > 0 (\delta$ 为常数), 若$\displaystyle \sum_{i=1}^{n} a_i = 1$, 证明: $$\sum_{i=1}^{n} \frac{i\sqrt[i]{a_1 a_2 \cdots a_i b_1 b_2 \cdots b_i}}{b_i b_{i+1}} \leq \frac{1}{\delta}.$$
			~\\

			$$i\sqrt[i]{a_1 a_2 \cdots a_i b_1 b_2 \cdots b_i} \leq a_1 b_1 + \cdots + a_i b_i,$$

			记

			$$S_i = a_1 b_1 + a_2 b_2 + \cdots + a_i b_i,$$

			又

			$$\frac{1}{b_i}-\frac{1}{b_{i+1}} = \frac{b_{i+1} - b_i}{b_i b_{i+1}} \geq \frac{\delta}{b_i b_{i+1}} \Rightarrow \frac{1}{b_i b_{i+1}} \leq \frac{1}{\delta}\left(\frac{1}{b_i} - \frac{1}{b_{i+1}}\right),$$

			有

			\begin{align*}
				\text{原式} &\leq \sum_{i=1}^{n} \frac{S_i}{b_i b_{i+1}}\\
				&\leq \sum_{i=1}^{n} \frac{1}{\delta} S_i \left(\frac{1}{b_i} - \frac{1}{b_{i+1}}\right)\\
				&= \frac{1}{\delta} \left(\sum_{i=2}^{n} \frac{S_i - S_{i-1}}{b_i} - \frac{S_n}{b_{n+1}} + \frac{S_1}{b_1}\right)\\
				&= \frac{1}{\delta} \left(\sum_{i=2}^{n} a_i - \frac{S_n}{b_{n+1}}+a_1\right)\\
				&= \frac{1}{\delta} \left(1-\frac{S_n}{b_{n+1}}\right)\\
				&< \frac{1}{\delta}.
			\end{align*}

		~\\

		\item 已知正数$x_1, x_2, \cdots, x_n$和$y_1, y_2, \cdots, y_n$满足$x_1 > x_2 > \cdots > x_n$, $y_1 > y_2 > \cdots > y_n$, 且$x_1 > y_1$, $x_1 + x_2 > y_1 + y_2$, $\cdots$, $x_1 + x_2 + \cdots + x_n > y_1 + y_2 + \cdots + y_n$. 求证: 对任意正整数$k$, 有$x_1^k + x_2^k + \cdots + x_n^k > y_1^k + y_2^k + \cdots + y_n^k.$
			~\\

			设$S_i = x_1 + \cdots + x_i, T_i = y_1 + \cdots + y_i, S_i > T_i,$

			\begin{align*}
				\sum_{i=1}^{n} (x_i^k - y_i^k) &= \sum_{i=1}^{n} (x_i - y_i) \left(x_i^{k-1} + x_i^{k-2} y_i + \cdots + y_i^{k-1}\right)\\
				&= \sum_{i=1}^{n-1} \left(S_i - T_i \right) \left(c_i - c_{i+1}\right) + (S_n - T_n)c_n > 0.
			\end{align*}

		~\\

		\item 设$x_i \geq 0 (i = 1, 2, \cdots, n)$, 且$$\sum_{i=1}^{n} x_i^2 + 2 \sum_{1\leq k<j\leq n} \sqrt{\frac{k}{j}} x_k x_j = 1,$$ 求$\displaystyle \sum_{i=1}^{n} x_i$的最大值和最小值.
			~\\

			$$\sqrt{\frac{k}{j}} < 1 \Rightarrow 1\leq \sum_{i=1}^{n} x_i^2 + 2\sum_{1\leq k<j\leq n} x_k x_j = \left(\sum_{i=1}^{n} x_i\right)^2 \Rightarrow \sum_{i=1}^{n} x_i \geq 1, x_1 = 1, x_2 = \cdots = x_n = 0.$$

			记$y_i = \frac{x_i}{\sqrt{i}}$, 则$x_i = \sqrt{i} y_i (y_i \geq 0),$

			$$\sum_{i=1}^{n} iy_i^2 + 2\sum_{1\leq k<j\leq n} k y_k \cdot y_j = 1,$$

			$$y_n^2 + (y_n + y_{n-1})^2 + \cdots + (y_n + y_{n-1} + \cdots + y_1)^2 = 1.$$

			记$S_i = y_n + \cdots + y_i$, 则$S_n^2 + \cdots + S_1^2 = 1,$ 显然有$S_1 \geq S_2 \geq \cdots \geq S_n, S_{n+1}=0,$

			\begin{align*}
				\sum_{i=1}^{n} x_i &= \sum_{i=1}^{n} \sqrt{i} \cdot \left(S_i - S_{i+1}\right)\\
				&= \sum_{i=2}^{n} S_i \left(\sqrt{i} - \sqrt{i-1}\right) + S_1 - \sqrt{n} S_{n+1}\\
				&= \sum_{i=1}^{n} S_i \left(\sqrt{i} - \sqrt{i-1}\right),
			\end{align*}

			\begin{align*}
				\left(\sum_{i=1}^{n} x_i\right)^2 &= \left[\sum_{i=1}^{n} S_i \left(\sqrt{i} - \sqrt{i-1}\right)\right]^2\\
				&\leq \left(\sum_{i=1}^{n} S_i^2 \right) \cdot \sum_{i=1}^{n} \left(\sqrt{i} - \sqrt{i-1}\right)^2 \tag*{\text{柯西不等式}}\\
				&= \sum_{i=1}^{n} \left(\sqrt{i} - \sqrt{i-1}\right)^2,
			\end{align*}

			等号成立条件

			$$\frac{S_1}{\sqrt{1}-\sqrt{0}} = \frac{S_2}{\sqrt{2}-\sqrt{1}} = \cdots = \frac{S_n}{\sqrt{n} - \sqrt{n-1}}.$$

		~\\

		\item 设$n\in\mathbb{N}^{*}, S\subseteq \left\{1, 2, \cdots, n\right\}, S\neq\emptyset$. 求证: $$\left(\sum_{i\in S} a_i\right)^2 \leq \sum_{1\leq i\leq j\leq n} (a_i + a_{i+1} + \cdots + a_j)^2. \eqno{(*)}$$
			~\\

			记$S_k = a_1 + a_2 + \cdots + a_k (k\in 1,2, \cdots, n), S_0 = 0$,

			\begin{align*}
				\text{右} &= \sum_{1\leq i\leq j\leq n}\left(S_j - S_{i-1}\right)^2\\
				&= \sum_{0\leq i<j\leq n}\left(S_j - S_i\right)^2\\
				&= (n+1) \sum_{i=0}^{n} S_i^2 - \left(\sum_{i=0}^{n} S_i\right)^2,
			\end{align*}

			$$
				\text{左} = \left[\sum_{i \in S} \left(S_i - S_{i-1} \right)\right]^2,
			$$

			$$
				\sum_{i \in S} \left(S_i - S_{i-1} \right) = \sum_{0\leq j\leq n} \lambda_i S_i (\lambda_i \in \{-1, 0, 1\}),
			$$

			$$
				\left(\sum_{0\leq j\leq n} \lambda_i S_i\right)^2 \leq \left(\sum_{0\leq i\leq n} \lambda_i \right)^2 \left(\sum_{0\leq i\leq n} S_i\right)^2 \leq (n+1) \sum_{i=0}^{n} S_i^2.
			$$

			不妨设$\displaystyle \sum_{i=0}^{n} S_i = 0$, 若$\displaystyle \sum_{i=0}^{n}S_i \neq 0$, 令$\displaystyle S_k'=S_k-\frac{\sum_{i=0}^{n} S_i}{n+1}$替换, 不等式等价.

	\end{enumerate}
\end{document}